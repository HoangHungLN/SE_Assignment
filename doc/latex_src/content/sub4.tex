\section{Working demonstration}

Nhóm đã xây dựng một bản demo hoạt động của hệ thống. Toàn bộ quá trình thao tác trên giao diện đã được ghi hình và lưu tại đường dẫn sau:

\url{https://drive.google.com/file/d/1LQ0IEKqISMwtK1SG1NGtqph8aZ6JW_VS/view?usp=sharing}


% Bạn có thể chỉnh sửa lại nội dung từng item cho khớp với video demo thực tế.

\section{How Generative AI was used}

Trong quá trình thực hiện bài tập lớn, nhóm có sử dụng một số công cụ Generative AI với phạm vi và mức độ đóng góp như sau:

\begin{itemize}
  \item \textbf{Phác thảo khung bài làm:} ChatGPT được sử dụng để phác thảo khung bài làm ban đầu cho từng lần nộp (submission). Dựa trên các gợi ý này, nhóm thống nhất các hạng mục công việc và phân chia nhiệm vụ cho từng thành viên.
  
  \item \textbf{Hỗ trợ lý thuyết:} ChatGPT và NotebookLM được dùng như công cụ tham khảo lý thuyết, giúp giải thích những phần kiến thức của môn học mà một số thành viên chưa nắm vững. Quyết định cuối cùng về nội dung lý thuyết do các thành viên tự tổng hợp và biên soạn.
  
  \item \textbf{Gợi ý yêu cầu phi chức năng và luồng thay thế:} Nhóm sử dụng ChatGPT và Gemini để gợi ý thêm Non-Functional Requirements và một số Alternative Flows trong Use-case Scenario. Từ các gợi ý này, nhóm tự rà soát, chọn lọc và chỉnh sửa cho phù hợp với bối cảnh hệ thống.
  
  \item \textbf{Gợi ý cho biểu đồ UML:} ChatGPT và Gemini cũng được dùng để tham khảo ý tưởng cho các Sequence Diagram và Activity Diagram. Việc phân tích hệ thống, xác định actor, luồng tương tác và chi tiết các bước được thực hiện trực tiếp bởi các thành viên nhóm.
  
  \item \textbf{Diễn đạt nội dung:} ChatGPT được dùng để hỗ trợ diễn đạt lại các ý tưởng do nhóm đề xuất theo hướng rõ ràng, logic, dễ hiểu hơn, đồng thời hạn chế lỗi ngữ pháp.
  
  \item \textbf{Hỗ trợ LaTeX:} GitHub Copilot được sử dụng để gợi ý cú pháp LaTeX (ví dụ: môi trường bảng, hình, tham chiếu chéo), giúp quá trình soạn thảo báo cáo diễn ra trôi chảy và hiệu quả hơn. Nội dung thực tế do các thành viên nhập và kiểm tra.
  
  \item \textbf{Rà soát lỗi:} ChatGPT, Claude và Gemini được nhóm sử dụng để rà soát, chỉ ra các điểm chưa hợp lý hoặc sai sót (nếu có) trong bài làm (yêu cầu, biểu đồ, bố cục báo cáo,...). Nhóm chủ động xem xét các phản hồi này và tự quyết định việc chỉnh sửa trước khi nộp.
\end{itemize}

Nhìn chung, các công cụ Generative AI chỉ đóng vai trò \emph{hỗ trợ tham khảo, gợi ý và rà soát}, còn mọi quyết định thiết kế, phân tích và nội dung cuối cùng đều do các thành viên trong nhóm tự chịu trách nhiệm và hoàn thiện.
