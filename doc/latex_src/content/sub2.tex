% ================== SUBMISSION 2 ==================
% ==================================================


\section{System Modelling}

% =======================UI design Mockup===========================
\subsection{UI design Mockup}

\subsubsection{Trang chủ và đăng nhập}

\begin{figure}[H]
  \centering
  % Ảnh trên cùng
  \begin{subfigure}[b]{0.48\textwidth}
    \centering
    \includegraphics[width=\textwidth]{graphics/sub2/UI_mockup/Trangchu/login.png}
    \caption{Trang đăng nhập}
  \end{subfigure}
  \begin{subfigure}[b]{0.48\textwidth}
    \centering
    \includegraphics[width=\textwidth]{graphics/sub2/UI_mockup/Trangchu/adminHomepage.png}
    \caption{Trang chủ admin}
  \end{subfigure}

  % \vspace{0.6em} % khoảng cách giữa hàng trên và hàng dưới

  % Hai ảnh hàng dưới
  \begin{subfigure}[b]{0.48\textwidth}
    \centering
    \includegraphics[width=\textwidth]{graphics/sub2/UI_mockup/Trangchu/studentHomepage.png}
    \caption{Trang chủ sinh viên}
  \end{subfigure}
  \hfill
  \begin{subfigure}[b]{0.48\textwidth}
    \centering
    \includegraphics[width=\textwidth]{graphics/sub2/UI_mockup/Trangchu/tutorHomepage.png}
    \caption{Trang chủ giảng viên}
  \end{subfigure}

  \caption{Giao diện trang chủ và đăng nhập.}
\end{figure}


% ===== Nhóm 1: 2 ảnh =====
\subsubsection{Sinh viên đăng ký nhóm và tutor nhận/từ chối yêu cầu hướng dẫn}
\begin{figure}[htbp]
  \centering
  \begin{subfigure}[b]{0.6\textwidth}
    \centering
    \includegraphics[width=\textwidth]{graphics/sub2/UI_mockup/Group/1SV_DKGroup_Main_v2.png}
    \caption{Sinh viên đăng ký nhóm}
  \end{subfigure}
  \hfill
  \begin{subfigure}[b]{0.6\textwidth}
    \centering
    \includegraphics[width=\textwidth]{graphics/sub2/UI_mockup/Group/1tutor_teachingRegister.png}
    \caption{Tutor tiếp nhận/ từ chối hướng dẫn}
  \end{subfigure}
  \caption{Các giao diện của chức năng đăng ký nhóm.}
\end{figure}

% ===== Nhóm 2: 3 ảnh =====
\subsubsection{Thông tin chi tiết nhóm}
\begin{figure}[H]
  \centering
  \begin{subfigure}[b]{0.48\textwidth}
    \centering
    \includegraphics[width=\textwidth]{graphics/sub2/UI_mockup/Group/2SV_DKGroup_GrDetail_v2.png}
    \caption*{Thông tin chi tiết nhóm}
  \end{subfigure}
  \hfill
\end{figure}

% % ===== Nhóm 3: 2 ảnh =====
\subsubsection{Thao tác đăng ký nhóm mới}

\begin{figure}[H]
  \begin{subfigure}[b]{0.45\textwidth}
    \centering
    \includegraphics[width=\textwidth]{graphics/sub2/UI_mockup/Group/2SV_DKGroup_NewReg.png}
    \caption{Form đăng ký mới}
  \end{subfigure}
  \hfill
  \begin{subfigure}[b]{0.45\textwidth}
    \centering
    \includegraphics[width=\textwidth]{graphics/sub2/UI_mockup/Group/3SV_DKGroup_TutorList.png}
    \caption{Danh sách tutor phù hợp}
  \end{subfigure}
\end{figure}

\begin{figure}[H]
  \centering
  \begin{subfigure}[b]{0.45\textwidth}
    \centering
    \includegraphics[width=\textwidth]{graphics/sub2/UI_mockup/Group/3SV_DKGroup_TutorInfor.png}
    \caption*{(c) Thông tin chi tiết giảng viên}
  \end{subfigure}
  \hfill
  \begin{subfigure}[b]{0.45\textwidth}
    \centering
    \includegraphics[width=\textwidth]{graphics/sub2/UI_mockup/Group/3SV_DKGroup_SubjectList.png}
    \caption*{(d) Danh sách môn tutor đăng ký dạy}
  \end{subfigure}
\end{figure}

\subsubsection{Tutor mở buổi học}
\begin{figure}[H]
  \centering
  \begin{minipage}[b]{0.45\textwidth}
    \centering
    % Ảnh lớn bên trái (dọc)
    \includegraphics[width=0.8\textwidth]{graphics/sub2/UI_mockup/buoihoc/Tutor_DkiLop_course.png}
    \caption{Đăng ký mở buổi học}
  \end{minipage}%
  \hfill
  \begin{minipage}[b]{0.55\textwidth}
    % Hai ảnh nhỏ xếp chồng bên phải
    \centering
    \includegraphics[width=\textwidth]{graphics/sub2/UI_mockup/buoihoc/Tutor_DkiLop_edit.png}
    \caption{Chỉnh sửa thông tin buổi học}

    \vspace{0.5em}

    \includegraphics[width=\textwidth]{graphics/sub2/UI_mockup/buoihoc/Tutor_DkiLop_main.png}
    \caption{Giao diện chính quản lý buổi học}
  \end{minipage}

  \caption{Giao diện quản lý buổi học của giảng viên.}
\end{figure}

\subsubsection{Sinh viên tham gia buổi học}

\begin{figure}[H]
  \centering

  % Hai ảnh phụ
  \begin{subfigure}[b]{\textwidth}
    \centering
    \includegraphics[width=\textwidth]{graphics/sub2/UI_mockup/buoihoc/Studen_DkiLop_main.png}
    \caption{Trang danh sách lớp học}
  \end{subfigure}
  \hfill

  \caption{Giao diện chức năng đăng ký buổi học của sinh viên.}
\end{figure}

% \subsubsection{Sinh viên truy cập buổi học}
% \begin{figure}[H]
%   \centering
%   \begin{subfigure}[b]{0.32\textwidth}
%     \includegraphics[width=\textwidth]{graphics/sub2/UI_mockup/class/Sv_danhsachbuoihoc.png}
%     \caption{Danh sách buổi học}
%   \end{subfigure}
%   \hfill
%   \begin{subfigure}[b]{0.32\textwidth}
%     \includegraphics[width=\textwidth]{graphics/sub2/UI_mockup/class/Sv_chitietbuoihoc.png}
%     \caption{Chi tiết buổi học}
%   \end{subfigure}
%   \hfill
%   \begin{subfigure}[b]{0.32\textwidth}
%     \includegraphics[width=\textwidth]{graphics/sub2/UI_mockup/class/Sv_PhanHoi.png}
%     \caption{Phản hồi của sinh viên}
%   \end{subfigure}
%   \caption{Các giao diện chức năng của sinh viên.}
% \end{figure}

% \subsubsection{Turtor truy cập buổi học}
% \begin{figure}[H]
%   \centering
%   % Ảnh hàng 1
%   \begin{subfigure}[b]{0.32\textwidth}
%     \includegraphics[width=\textwidth]{graphics/sub2/UI_mockup/class/Tutor_danhsachbuoihoc.png}
%     \caption{Danh sách buổi học}
%   \end{subfigure}
%   \hfill
%   \begin{subfigure}[b]{0.32\textwidth}
%     \includegraphics[width=\textwidth]{graphics/sub2/UI_mockup/class/Tutor_chitietbuoihoc.png}
%     \caption{Chi tiết buổi học}
%   \end{subfigure}
%   \hfill
%   \begin{subfigure}[b]{0.32\textwidth}
%     \includegraphics[width=\textwidth]{graphics/sub2/UI_mockup/class/Tutor_DanhSachsinhvien.png}
%     \caption{Danh sách sinh viên tham gia}
%   \end{subfigure}
%   \caption{Các giao diện quản lý buổi học của Tutor.}
% \end{figure}

% \subsubsection{Tutor tổ chức và đánh giá buổi học}
% \begin{figure}[H]
%   \centering
%   \begin{subfigure}[b]{0.48\textwidth}
%     \includegraphics[width=\textwidth]{graphics/sub2/UI_mockup/class/Tutor_ToChuc_ChiTiet_DiemDanh.png}
%     \caption{Chi tiết điểm danh buổi học}
%   \end{subfigure}
%   \hfill
%   \begin{subfigure}[b]{0.48\textwidth}
%     \includegraphics[width=\textwidth]{graphics/sub2/UI_mockup/class/Tutor_ToChuc_ChiTiet_DanhGia.png}
%     \caption{Chi tiết đánh giá buổi học}
%   \end{subfigure}
%   \caption{Các giao diện tổ chức và đánh giá buổi học.}
% \end{figure}

\subsubsection{Sinh viên - Buổi học của tôi}
\begin{figure}[H]
  \centering
  \begin{subfigure}[b]{\textwidth}
    \includegraphics[width=\textwidth]{graphics/sub2/UI_mockup/Student-Buoi hoc cua toi/01.Student-BuoihocCuaToi.jpeg}
    \caption{Trang Buổi học của tôi}
  \end{subfigure}

  \vspace{0.6em}
  
  \begin{subfigure}[b]{0.48\textwidth}
    \includegraphics[width=\textwidth]{graphics/sub2/UI_mockup/Student-Buoi hoc cua toi/02.Student-ChiTietTungBuoi.png}
    \caption{Chi tiết từng buổi}
  \end{subfigure} \hfill
  \begin{subfigure}[b]{0.48\textwidth}
    \includegraphics[width=\textwidth]{graphics/sub2/UI_mockup/Student-Buoi hoc cua toi/03.Student-PhanHoiBuoiHoc.png}
    \caption{Phản hồi buổi học}
  \end{subfigure}

\end{figure}

\subsubsection{Tutor - Quản lý buổi học}
\begin{figure}[H]
  \centering
  \begin{subfigure}[b]{0.95\textwidth}
    \includegraphics[width=\textwidth]{graphics/sub2/UI_mockup/Tutor-Quan ly buoi hoc/01.Tutor-QuanLyBuoiHoc.png}
    \caption{Trang Quản lý buổi học}
  \end{subfigure}

  \vspace{0.6em}
  
  \begin{subfigure}[b]{0.95\textwidth}
    \includegraphics[width=\textwidth]{graphics/sub2/UI_mockup/Tutor-Quan ly buoi hoc/02.Tutor-ChiTietTungBuoi.png}
    \caption{Chi tiết từng buổi}
  \end{subfigure}
\end{figure}

\begin{figure}[H]
  \centering
  \begin{subfigure}[b]{0.95\textwidth}
    \centering
    \includegraphics[width=\textwidth]{graphics/sub2/UI_mockup/Tutor-Quan ly buoi hoc/03.Tutor-QuanLyTaiLieu.png}
    \caption*{(c) Quản lý tài liệu}
  \end{subfigure}\hfill

  \vspace{0.6em}
  
  \begin{subfigure}[b]{0.95\textwidth}
    \centering
    \includegraphics[width=\textwidth]{graphics/sub2/UI_mockup/Tutor-Quan ly buoi hoc/04.Tutor-DongBoTaiLieu.png}
    \caption*{(d) Đồng bộ tài liệu}
  \end{subfigure}
\end{figure}

\begin{figure}[H]
  \centering
  \begin{subfigure}[b]{0.95\textwidth}
    \centering
    \includegraphics[width=\textwidth]{graphics/sub2/UI_mockup/Tutor-Quan ly buoi hoc/05.Tutor-DanhGiaTienDoSinhVien.png}
    \caption*{(e) Đánh giá tiến độ sinh viên}
  \end{subfigure}\hfill

  \vspace{0.6em}

  \begin{subfigure}[b]{0.95\textwidth}
    \centering
    \includegraphics[width=\textwidth]{graphics/sub2/UI_mockup/Tutor-Quan ly buoi hoc/06.Tutor-DiemDanhSinhVien.png}
    \caption*{(f) Điểm danh sinh viên}
  \end{subfigure}
\end{figure}

\subsubsection{Phòng Công tác Sinh viên}
\begin{figure}[H]
  \centering
  % Ảnh trên cùng
  \begin{subfigure}[b]{0.6\textwidth}
    \centering
    \includegraphics[width=\textwidth]{graphics/sub2/UI_mockup/Offices/ctsv_xemthamgia.png}
    \caption{Kết quả tham gia của sinh viên}
  \end{subfigure}

  \vspace{0.6em}

  % Hai ảnh hàng dưới
  \begin{subfigure}[b]{0.47\textwidth}
    \centering
    \includegraphics[width=\textwidth]{graphics/sub2/UI_mockup/Offices/ctsv_xemTutor_00.png}
    \caption{Các nhóm của tutor}
  \end{subfigure}\hfill
  \begin{subfigure}[b]{0.47\textwidth}
    \centering
    \includegraphics[width=\textwidth]{graphics/sub2/UI_mockup/Offices/ctsv_xemTutor_01.png}
    \caption{Tutor đánh giá sinh viên}
  \end{subfigure}

  \caption{Phòng CTSV xem báo cáo.}
\end{figure}

\subsubsection{Khoa/Bộ môn}
\begin{figure}[H]
  \centering
  % Hàng 1
  \begin{subfigure}[b]{0.48\textwidth}
    \centering
    \includegraphics[width=\textwidth]{graphics/sub2/UI_mockup/Offices/khoa_xemSinhvien_00.png}
    \caption{Phản hồi của sinh viên (1)}
  \end{subfigure}\hfill
  \begin{subfigure}[b]{0.48\textwidth}
    \centering
    \includegraphics[width=\textwidth]{graphics/sub2/UI_mockup/Offices/khoa_xemSinhvien_01.png}
    \caption{Phản hồi của sinh viên (2)}
  \end{subfigure}

  \vspace{0.6em}

  % Hàng 2
  \begin{subfigure}[b]{0.48\textwidth}
    \centering
    \includegraphics[width=\textwidth]{graphics/sub2/UI_mockup/Offices/khoa_xemTutor_00.png}
    \caption{Các nhóm của tutor}
  \end{subfigure}\hfill
  \begin{subfigure}[b]{0.48\textwidth}
    \centering
    \includegraphics[width=\textwidth]{graphics/sub2/UI_mockup/Offices/khoa_xemTutor_01.png}
    \caption{Tutor đánh giá sinh viên}
  \end{subfigure}

  \caption{Khoa/Bộ môn xem báo cáo.}
\end{figure}

\subsubsection{Phòng Đào tạo \& Điều phối viên}
\begin{figure}[H]
  \centering
  % Hàng 1
  \begin{subfigure}[b]{0.48\textwidth}
    \centering
    \includegraphics[width=\textwidth]{graphics/sub2/UI_mockup/Offices/pdt_xembaocao.png}
    \caption{Báo cáo tổng quan (PĐT)}
  \end{subfigure}\hfill
  \begin{subfigure}[b]{0.48\textwidth}
    \centering
    \includegraphics[width=\textwidth]{graphics/sub2/UI_mockup/Offices/dieuphoi_xembaocao.png}
    \caption{Báo cáo tổng quan (Điều phối viên)}
  \end{subfigure}

  \vspace{0.6em}

  % Hàng 2
  \begin{subfigure}[b]{0.48\textwidth}
    \centering
    \includegraphics[width=\textwidth]{graphics/sub2/UI_mockup/Offices/pdt_xemSinhvien_00.png}
    \caption{Phản hồi sinh viên (1)}
  \end{subfigure}\hfill
  \begin{subfigure}[b]{0.48\textwidth}
    \centering
    \includegraphics[width=\textwidth]{graphics/sub2/UI_mockup/Offices/pdt_xemSinhvien_01.png}
    \caption{Phản hồi sinh viên (2)}
  \end{subfigure}

  \caption{Chức năng xem báo cáo và phản hồi của PĐT \& Điều phối viên.}
\end{figure}


% =======================Sequence diagrams===========================
\subsection{Sequence diagrams}
\subsubsection{Sinh viên đăng ký nhóm:}
\begin{figure}[H]
  \centering
  \adjincludegraphics[
    width=\linewidth,
    trim={{0\width} {0\height} {0\width} {0\height}},
    clip
  ]{graphics/sub2/Sequence/Seq_GrRegister.png} % ← thay bằng đường dẫn thực tế của ảnh
  \caption{Sequence diagram: Sinh viên đăng ký nhóm}   % ← thay caption của ảnh
\end{figure}
\pagebreak
\textbf{Mô tả sequence diagram: Sinh viên đăng ký nhóm mới} \\
\begin{itemize}
    \item Diagram có bốn lifelines:
    \begin{itemize}
        \item Actor: Sinh viên, người thao tác đăng ký nhóm.
        \item Boundary SystemInterface (SI): giao diện và điểm tiếp nhận thao tác.
        \item Control GroupController (GC): xử lý logic cho quá trình đăng ký nhóm.
        \item Entity GroupDatabase (GD): lưu trữ dữ liệu nhóm.
    \end{itemize}

    \item Khi sinh viên nhấn nút đăng ký nhóm, SI nhận tín hiệu và hiển thị bảng để nhập thông tin.

    \item Trường hợp sinh viên gửi thông tin:
    \begin{itemize}
        \item SI gửi dữ liệu sang GC để yêu cầu tạo nhóm mới.
        \item GC tạo đối tượng nhóm với thuộc tính do sinh viên nhập, đặt trạng thái “đang chờ hướng dẫn”.
        \item GC yêu cầu GD lưu nhóm vào cơ sở dữ liệu.
        \item GD lưu dữ liệu:
        \begin{itemize}
            \item Nếu lỗi, GD phản hồi cho GC; GC báo lại SI; SI hiển thị thông báo tạo nhóm không thành công.
            \item Nếu lưu thành công, GC báo lại SI để hiển thị thông báo tạo nhóm thành công.
        \end{itemize}
    \end{itemize}

    \item Trường hợp sinh viên nhấn hủy:
    \begin{itemize}
        \item SI nhận yêu cầu và kết thúc thao tác mà không gửi dữ liệu sang GC.
    \end{itemize}
\end{itemize}

\pagebreak
\subsubsection{Tutor chấp nhận/từ chối nhóm:}
\begin{figure}[H]
  \centering
  \adjincludegraphics[
    width=0.8\linewidth,
    trim={{0\width} {0\height} {0\width} {0\height}},
    clip
  ]{graphics/sub2/Sequence/Seq_TutorChapNhanTuChoiNhom.png} % ← thay bằng đường dẫn thực tế của ảnh
  \caption{Sequence diagram: Tutor chấp nhận/từ chối nhóm}   % ← thay caption của ảnh
\end{figure}

\pagebreak
\textbf{Mô tả sequence diagram: Tutor chấp nhận/từ chối nhóm} \\

Diagram gồm 5 lifelines bao gồm \textit{Actor} là Tutor, \textit{Boundary} là 
\textit{SystemInterface} (SI) đóng gói lại hệ thống và cung cấp phương thức cho 
người dùng tương tác, với bộ phận \textit{Control} điều khiển logic cho quá trình 
là \textit{GroupController} (GC), và hệ thống dữ liệu \textit{Entity} là 
\textit{GroupDatabase} (GD) chịu trách nhiệm lưu trữ dữ liệu nhóm.

Khi Tutor truy cập vào danh sách nhóm yêu cầu, SI sẽ gửi yêu cầu đến GC để lấy 
danh sách. GC sẽ yêu cầu GD trả về danh sách các nhóm có trạng thái ``chờ xử lý''.
Kết quả trả về sẽ có hai trường hợp:

\begin{itemize}
    \item \textbf{Trường hợp 1:} Số lượng nhóm $=0$, SI sẽ hiển thị thông báo 
    ``Không có nhóm yêu cầu mới''.

    \item \textbf{Trường hợp 2:} Số lượng nhóm $>0$, SI sẽ hiển thị danh sách các 
    yêu cầu chờ xử lý. Tutor có thể chọn một yêu cầu để xem chi tiết thông tin nhóm. 
    Khi đó, SI sẽ yêu cầu GC lấy thông tin chi tiết của nhóm đó từ GD và hiển thị lên 
    cho Tutor.
\end{itemize}

Sau khi xem xét, Tutor có hai lựa chọn:

\begin{itemize}
    \item \textbf{Chấp nhận:} Tutor chọn ``Chấp nhận''. SI sẽ gửi yêu cầu xử lý 
    chấp nhận đến GC. GC sẽ cập nhật trạng thái của nhóm trong GD (thành ``Đã chấp nhận''). 
    Khi các thao tác này hoàn tất, SI sẽ hiển thị thông báo ``Chấp nhận thành công'' cho Tutor.

    \item \textbf{Từ chối:} Tutor chọn ``Từ chối''. SI sẽ gửi yêu cầu xử lý từ chối đến GC. 
    GC sẽ cập nhật trạng thái nhóm trong GD (thành ``Đã từ chối''). Khi hoàn tất, 
    SI sẽ hiển thị thông báo ``Từ chối thành công'' cho Tutor.
\end{itemize}

\pagebreak
\subsubsection{Tutor mở đăng ký dạy:}
\begin{figure}[H]
  \centering
  \adjincludegraphics[
    width=\linewidth,
    trim={{0\width} {0\height} {0\width} {0\height}},
    clip
  ]{graphics/sub2/Sequence/Seq_tutordangkyday_v2.png} % ← thay bằng đường dẫn thực tế của ảnh
  \caption{Sequence diagram: Tutor đăng ký dạy}   % ← thay caption của ảnh
\end{figure}

\pagebreak
\textbf{Mô tả sequence diagram: Tutor đăng ký dạy} \\
\begin{itemize}
    \item Diagram gồm bốn lifelines:
    \begin{itemize}
        \item Actor: Tutor, người thao tác đăng ký buổi dạy.
        \item Boundary SystemInterface (SI): tiếp nhận thao tác và hiển thị giao diện.
        \item Control SessionController (SC): xử lý toàn bộ logic tạo buổi học.
        \item Entity SessionDatabase (SD): lưu trữ dữ liệu buổi học.
    \end{itemize}

    \item Khi tutor chọn “Đăng ký dạy”, SI hiển thị bảng nhập thông tin.
    \item Khi tutor nhấn “Lưu buổi học”, SI gửi dữ liệu sang SC để yêu cầu tạo buổi học mới.
    \item SC tạo đối tượng buổi học, đặt trạng thái “Có thể đăng ký”, và yêu cầu SD lưu vào cơ sở dữ liệu.

    \item SD xử lý yêu cầu lưu:
    \begin{itemize}
        \item Nếu lỗi, SD phản hồi cho SC; SC báo về SI; SI thông báo đăng ký dạy không thành công.
        \item Nếu lưu thành công, SD phản hồi cho SC; SC gửi lại SI; SI hiển thị thông báo đăng ký dạy thành công.
    \end{itemize}
\end{itemize}

\subsubsection{Sinh viên tham gia buổi học:}
\begin{figure}[H]
  \centering
  \adjincludegraphics[
    width=0.9\linewidth,
    trim={{0\width} {0\height} {0\width} {0\height}},
    clip
  ]{graphics/sub2/Sequence/Seq_SinhvienNhanLop.png} % ← thay bằng đường dẫn thực tế của ảnh
  \caption{Sequence diagram: Sinh viên tham gia buổi học}   % ← thay caption của ảnh
\end{figure}

\pagebreak
\textbf{Mô tả sequence diagram: Sinh viên đăng ký tham gia buổi học} \\
\begin{itemize}
    \item Sequence diagram gồm bốn lifelines:
    \begin{itemize}
        \item Actor Sinh viên: người trực tiếp thao tác nhận buổi học.
        \item Boundary SystemInterface (SI): hiển thị giao diện và cung cấp chức năng tương tác.
        \item Control SessionController (SC): xử lý nghiệp vụ lấy danh sách và đăng ký buổi học.
        \item Entity SessionDatabase (SD): truy xuất và lưu trữ dữ liệu buổi học và đăng ký.
    \end{itemize}

    \item Sau khi sinh viên đăng nhập, SI hiển thị danh sách buổi học.
    \item Khi sinh viên truy cập trang “Đăng ký học”, SI gửi yêu cầu đến SC để lấy danh sách các buổi học khả dụng.
    \item SC yêu cầu SD truy xuất dữ liệu tất cả buổi học, nhận phản hồi, xử lý thông tin và trả kết quả về SI.
    \item Dựa vào trạng thái từng buổi học, SC gửi thông báo phù hợp để SI hiển thị danh sách và các nút thao tác tương ứng.
    
    \item Khi sinh viên nhấn “Tham gia”, SI gửi yêu cầu đăng ký tới SC.
    \item SC kiểm tra dữ liệu buổi học từ SD và xử lý theo tình trạng:
    \begin{itemize}
        \item Trùng lịch: SC báo lỗi để SI hiển thị thông báo.
        \item Đăng ký thành công: SC báo thành công và SD lưu thông tin đăng ký.
    \end{itemize}
\end{itemize}


\subsubsection{Sinh viên truy cập tài liệu:}
\begin{figure}[H]
  \centering
  \adjincludegraphics[
    width=0.75\linewidth,
    trim={{0\width} {0\height} {0\width} {0\height}},
    clip
  ]{graphics/sub2/Sequence/Seq_truycaptailieu.png} % ← thay bằng đường dẫn thực tế của ảnh
  \caption{Sequence diagram: Truy cập tài liệu}   % ← thay caption của ảnh
\end{figure}

\pagebreak
\textbf{Mô tả sequence diagram: Sinh viên truy cập tài liệu} \\
\begin{itemize}
    \item Người dùng chọn chức năng “Truy cập tài liệu”. SystemInterface (SI) nhận và chuyển yêu cầu đến LearningController (LC).
    \item LC truy vấn LearningDatabase (DB) để lấy danh sách tài liệu, trả về cho SI. SI hiển thị danh sách nếu không rỗng, ngược lại thông báo danh sách trống.
    \item Khi người dùng chọn một tài liệu và chế độ thao tác (xem hoặc tải xuống), SI gửi yêu cầu mở tài liệu đến LC.
    \item LC truy vấn DB để lấy thông tin meta, gồm tồn tại và quyền truy cập của tài liệu.
    \item Nếu tài liệu khả dụng:
    \begin{itemize}
        \item Với chế độ View: LC yêu cầu DB stream nội dung và trả luồng dữ liệu để người dùng xem trực tiếp.
        \item Với chế độ Download: LC lấy thông tin cho phép tải và kích hoạt quá trình tải xuống trên SI.
    \end{itemize}
    \item LC đồng thời ghi log truy cập (user, tài liệu, thời điểm) xuống DB theo cơ chế bất đồng bộ.
    \item Nếu tài liệu không khả dụng: LC báo lỗi để SI hiển thị thông báo “Tài liệu không tồn tại”.
\end{itemize}


% =======================Activity diagrams===========================

\subsection{Activity diagrams}
\subsubsection{Sinh viên đăng ký nhóm:}
\begin{figure}[H]
  \centering
  \adjincludegraphics[
    width=\linewidth,
    trim={{0\width} {0\height} {0\width} {0\height}},
    clip
  ]{graphics/sub2/activity/Ac_GrRegister.png} % ← thay bằng đường dẫn thực tế của ảnh
  \caption{Activity diagram: Sinh viên đăng ký nhóm}   % ← thay caption của ảnh
\end{figure}

\pagebreak
\textbf{Mô tả Activity diagram: Sinh viên đăng ký nhóm} \\
\begin{itemize}
    \item Diagram mô tả các bước sinh viên và hệ thống thực hiện khi đăng ký nhóm mới.
    \item Sinh viên nhấn nút “Đăng ký nhóm” và nhập thông tin nhóm.
    \item Nếu sinh viên đổi ý và nhấn “Hủy”, hệ thống dừng quy trình và không xử lý thêm.
    \item Nếu sinh viên nhấn “Gửi”, hệ thống nhận dữ liệu và kiểm tra tính đầy đủ:
    \begin{itemize}
        \item Nếu thông tin thiếu, hệ thống yêu cầu sinh viên nhập lại.
        \item Nếu thông tin đầy đủ, hệ thống tạo đối tượng “Nhóm” và đặt trạng thái “Đang chờ hướng dẫn”.
    \end{itemize}
    \item Hệ thống lưu nhóm vừa tạo vào cơ sở dữ liệu.
    \item Sau khi lưu:
    \begin{itemize}
        \item Nếu lưu thất bại, hệ thống thông báo lỗi cho sinh viên.
        \item Nếu lưu thành công, hệ thống thông báo tạo nhóm thành công.
    \end{itemize}
    \item Quy trình kết thúc.
\end{itemize}

\subsubsection{Tutor chấp nhận/từ chối nhóm yêu cầu:}
\begin{figure}[H]
  \centering
  \adjincludegraphics[
    width=\linewidth,
    trim={{0\width} {0\height} {0\width} {0\height}},
    clip
  ]{graphics/sub2/activity/Ac_TutorChapNhanTuChoiNhom.png} % ← thay bằng đường dẫn thực tế của ảnh
  \caption{Activity diagram: Tutor chấp nhận/từ chối nhóm yêu cầu}   % ← thay caption của ảnh
\end{figure}

\textbf{Mô tả Activity diagram: Tutor chấp nhận/từ chối nhóm yêu cầu:} \\
Diagram mô tả các hành động mà Tutor và hệ thống thực hiện trong quá trình xử lý
yêu cầu đăng ký nhóm với 2 swimlane: Tutor và hệ thống. Để bắt đầu, Tutor sẽ chọn
chức năng ``Xem danh sách nhóm yêu cầu''. Hệ thống sau đó sẽ kiểm tra và lấy danh
sách các nhóm đang chờ phê duyệt.

\begin{itemize}
    \item Nếu không có nhóm nào (Số lượng = 0), hệ thống sẽ hiển thị thông báo 
    ``Không có nhóm yêu cầu mới'' và kết thúc luồng xử lý.

    \item Nếu có ít nhất một nhóm (Số lượng > 0), hệ thống sẽ hiển thị danh sách 
    các yêu cầu này. Tutor có thể chọn một yêu cầu để ``Xem thông tin nhóm chi tiết''. 
    Sau khi xem xét, Tutor sẽ đưa ra quyết định.
\end{itemize}

Tutor có hai lựa chọn:

\begin{itemize}
    \item \textbf{Chấp nhận:} Tutor chọn ``Chấp nhận''. Hệ thống sẽ thực hiện chuỗi 
    hành động: cập nhật trạng thái nhóm thành ``Đã chấp nhận''. Cuối cùng, hệ thống 
    hiển thị kết quả ``Xác nhận chấp nhận thành công'' cho Tutor.

    \item \textbf{Từ chối:} Tutor chọn ``Từ chối''. Hệ thống sẽ thực hiện chuỗi 
    hành động: cập nhật trạng thái nhóm thành ``Đã từ chối''. Cuối cùng, hệ thống 
    hiển thị thông báo ``Từ chối thành công'' cho Tutor.
\end{itemize}

Sau khi mọi hành động được hoàn tất, quy trình sẽ kết thúc.



\subsubsection{Tutor đăng ký dạy:}
\begin{figure}[H]
  \centering
  \adjincludegraphics[
    width=\linewidth,
    trim={{0\width} {0\height} {0\width} {0\height}},
    clip
  ]{graphics/sub2/activity/Ac_tutordangkyday_v2.png} % ← thay bằng đường dẫn thực tế của ảnh
  \caption{Activity diagram: Tutor đăng ký dạy}   % ← thay caption của ảnh
\end{figure}

\textbf{Mô tả Activity diagram: Tutor đăng ký dạy} 

\begin{itemize}
    \item Tutor chọn chức năng “Đăng ký dạy”, nhập thông tin buổi dạy và nhấn “Lưu buổi học” nếu muốn gửi yêu cầu.
    \item Hệ thống kiểm tra dữ liệu. Nếu thiếu, hệ thống yêu cầu nhập lại. Nếu hợp lệ, hệ thống tạo đối tượng “Buổi học”, đặt trạng thái “Có thể đăng ký”, và lưu vào danh sách buổi học khả dụng cùng danh sách buổi dạy của tutor.
    \item Hệ thống thông báo kết quả tùy theo việc lưu dữ liệu thành công hoặc thất bại và kết thúc quy trình.
\end{itemize}

\pagebreak
\subsubsection{Sinh viên tham gia buổi học:}
\begin{figure}[H]
  \centering
  \adjincludegraphics[
    width=\linewidth,
    trim={{0\width} {0\height} {0\width} {0\height}},
    clip
  ]{graphics/sub2/activity/Ac_SinhvienNhanLop.png} % ← thay bằng đường dẫn thực tế của ảnh
  \caption{Activity diagram: Sinh viên tham gia buổi học}   % ← thay caption của ảnh
\end{figure}

\pagebreak
\textbf{Mô tả Activity diagram: Sinh viên tham gia buổi học} \\

\begin{itemize}
  \item Activity diagram mô tả chuỗi hoạt động giữa sinh viên và hệ thống trong quá trình sinh viên xem danh sách và đăng ký một buổi học. Ban đầu, sinh viên truy cập vào trang đăng ký học. Hệ thống tiếp nhận yêu cầu và tiến hành truy vấn cơ sở dữ liệu để lấy thông tin các buổi học hiện có, sau đó trả kết quả về cho giao diện để hiển thị theo từng trạng thái như: buổi học đã đăng ký, buổi học chưa đăng ký hoặc buổi học bị trùng lịch.
  \item Khi đã xem danh sách, sinh viên có thể lựa chọn một buổi học để đăng ký. Nếu không chọn buổi học nào, tiến trình kết thúc ngay tại giao diện. Trường hợp sinh viên chọn buổi học để đăng ký, hệ thống sẽ gửi yêu cầu kiểm tra thông tin đăng ký. Nếu bị trùng lịch với buổi đăng kí trước đó thì hệ thống sẽ xử lý và trả về “không thể đăng kí do trùng lịch”, sau đó giao diện hiển thị lỗi và kết thúc hành động. Ngược lại nếu không trùng lịch thì hệ thống gửi yêu cầu đăng ký vào hệ cơ sở dữ liệu, sau khi lưu thành công, hệ thống phải hồi lại và giao diện cập nhập trạng thái buổi học và gửi thông báo “Đăng ký thành công” cho sinh viên. 
\end{itemize}

\pagebreak
\subsubsection{Sinh viên truy cập tài liệu:}
\begin{figure}[H]
  \centering
  \adjincludegraphics[
    width=0.8\linewidth,
    trim={{0\width} {0\height} {0\width} {0\height}},
    clip
  ]{graphics/sub2/activity/Ac_truycaptailieu.png} % ← thay bằng đường dẫn thực tế của ảnh
  \caption{Activity diagram: Truy cập tài liệu}   % ← thay caption của ảnh
\end{figure}

\pagebreak
\textbf{Mô tả Activity diagram: Truy cập tài liệu} 

\begin{itemize}
  \item Activity diagram mô tả quy trình người dùng truy cập và xem tài liệu trong hệ thống. Quy trình bắt đầu khi người dùng chọn chức năng “Xem danh sách tài liệu”. Hệ thống tiếp nhận yêu cầu và thực hiện bước lấy danh sách tài liệu từ kho dữ liệu. Tại đây, hệ thống kiểm tra xem danh sách có trống hay không. Nếu không có tài liệu, hệ thống hiển thị thông báo danh sách rỗng và kết thúc quy trình. Nếu có dữ liệu, hệ thống hiển thị danh sách tài liệu cho người dùng.
  \item Người dùng sau đó chọn một tài liệu cụ thể để truy cập, người dùng chọn xem hoặc tải tài liệu. Yêu cầu này được gửi đến hệ thống, hệ thống tiến hành kiểm tra tính khả dụng của tài liệu. Trong trường hợp tài liệu không khả dụng (ví dụ: bị xóa, hết quyền truy cập, lỗi lưu trữ), hệ thống hiển thị thông báo lỗi “Tài liệu không khả dụng” và kết thúc tiến trình. Nếu tài liệu khả dụng, hệ thống cho phép người dùng tiếp tục. Nếu người dùng chọn xem, hệ thống thực hiện stream tài liệu và hiển thị nội dung. Nếu người dùng chọn tải xuống, hệ thống kích hoạt chức năng tải file về thiết bị. Sau khi hoàn tất việc xem hoặc tải, hệ thống tiến hành bước cuối cùng là ghi nhận lại lịch sử truy cập, bao gồm thông tin về loại thao tác và thời điểm truy cập.
\end{itemize}

% =======================State-chart diagrams===========================
\pagebreak
\subsection{State-chart diagrams }
\subsubsection*{State-chart diagram của đối tượng buổi học:}
\begin{figure}[H]
  \centering
  \adjincludegraphics[
    width=\linewidth,
    trim={{0\width} {0\height} {0\width} {0\height}},
    clip
  ]{graphics/sub2/state_diagram.png}     % ← thay bằng đường dẫn thực tế của ảnh
  \caption{State diagram của đối tượng buổi học}   % ← thay caption của ảnh
\end{figure}
\pagebreak
\textbf{Mô tả  state diagram của đối tượng buổi học} \\

\begin{table}[h]
    \centering
    \begin{tabular}{|l|p{12cm}|}
        \hline
        \centering \textbf{State} & \multicolumn{1}{c|}{\textbf{Mô tả}} \\
        \hline
        \textbf{Khởi tạo} & Buổi học được tạo và xuất hiện trong danh sách của tutor. Tutor có thể kiểm tra và cập nhật các thông tin của buổi học trước khi mở đăng ký. \\
        \hline
        \textbf{Mở đăng ký} & Buổi học được hiển thị công khai cho sinh viên; hệ thống cho phép sinh viên đăng ký / hủy đăng ký và liên tục cập nhật số lượng sinh viên đã đăng ký. \\
        \hline
        \textbf{Đóng đăng ký} & Buổi học vẫn được hiển thị công khai. Hệ thống ngừng nhận đăng ký mới (chỉ cho phép hủy). Số lượng sinh viên vẫn được cập nhật khi có sinh viên hủy tham gia và khi rời state thì danh sách sinh viên được khóa. \\
        \hline
        \textbf{Chờ diễn ra} & Danh sách sinh viên đã được chốt, buổi học chờ tới giờ bắt đầu. Hệ thống gửi thông báo nhắc nhở đến các sinh viên đã đăng ký. \\
        \hline
        \textbf{Đang diễn ra} & Buổi học đang được tổ chức. Hệ thống hỗ trợ điểm danh sinh viên tham gia và khi kết thúc state sẽ lưu lại kết quả điểm danh. \\
        \hline
        \textbf{Hoàn thành} & Buổi học đã kết thúc. Hệ thống cho phép ghi nhận phản hồi, đánh giá từ tutor/sinh viên và lưu trữ toàn bộ thông tin liên quan đến buổi học. \\
        \hline
    \end{tabular}
\end{table}

\begin{table}[h]
    \centering
    % Sử dụng p{...} cho cả 2 cột để văn bản tự xuống dòng khi quá dài
    \begin{tabular}{|l|p{7.6cm}|}
        \hline
        \centering \textbf{Stimulus} & \multicolumn{1}{c|}{\textbf{Mô tả}} \\
        \hline
        \textbf{Tutor chỉnh sửa thông tin} & Tutor thay đổi hoặc cập nhật lại các thông tin chi tiết của buổi học \\
        \hline
        \textbf{Tutor mở đăng ký} & Tutor quyết định mở form đăng ký cho buổi học \\
        \hline
        \textbf{Sinh viên tham gia / Hủy tham gia} & Một sinh viên mới đăng ký tham gia buổi học hoặc hủy đăng ký \\
        \hline
        \textbf{Tutor đóng đăng ký} & Trong trạng thái “khởi tạo”, Tutor chủ động đóng form đăng ký. \\
        \hline
        \textbf{Tutor đóng đăng ký / Đủ số lượng} & Trong trạng thái “Mở đăng ký”, tutor thực hiện thao tác đóng đăng ký. \\
        \hline
        \textbf{Sinh viên hủy tham gia} & Khi đã ở trạng thái “Đóng đăng ký”, một sinh viên hủy tham gia \\
        \hline
        \textbf{Gần đến thời gian bắt đầu} & Thời điểm bắt đầu buổi học sắp đến \\
        \hline
        \textbf{Đến hạn bắt đầu} & Đến đúng giờ bắt đầu buổi học \\
        \hline
        \textbf{Hết thời gian buổi học} & Thời lượng buổi học kết thúc \\
        \hline
        \textbf{Tutor hủy} & Tutor quyết định hủy buổi học tại bất kỳ giai đoạn nào (Khởi tạo, Mở/Đóng đăng ký, Chờ diễn ra, Đang diễn ra) \\
        \hline
    \end{tabular}
\end{table}
\pagebreak